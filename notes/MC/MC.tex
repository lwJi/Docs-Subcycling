\documentclass[prd,aps,a4paper,superscriptaddress,onecolumn,footinbib]{revtex4}
\usepackage{graphicx}
\usepackage{color}
\usepackage{dcolumn}
\usepackage{bm}
\usepackage{upgreek}
\usepackage{slashed}
\usepackage{amsmath}
\usepackage{empheq}
\usepackage{latexsym}
\usepackage{amssymb}
\usepackage{amsfonts}
\usepackage{dsfont}
\usepackage{listings}
\usepackage{xcolor}
\usepackage{ulem}
\usepackage{cancel}
\usepackage{mathtools}
\usepackage{enumitem}
\setcounter{MaxMatrixCols}{18}



%%
\newcommand{\mytext}{}
\newcommand{\LJ}[1]{{\textcolor{blue}{\mytext{LJ: #1}} }}
\newcommand{\WT}[1]{{\textcolor{red}{\mytext{WT: #1}} }}
%%



\begin{document}
\title{Notes on Implementing MC's Method}

\author{Liwei~Ji}
\email{ljsma@rit.edu}
\affiliation{Center for Computational Relativity and Gravitation,
Rochester Institute of Technology, Rochester, New York 14623, USA}

\maketitle

\tableofcontents

\lstset{numbers=left,
  numberstyle= \tiny,
  keywordstyle= \color{ blue!70},commentstyle=\color{red!50!green!50!blue!50},
  frame=shadowbox,
  rulesepcolor= \color{ red!20!green!20!blue!20}
}


%%%%%%%%%%%%%%%%%%%%%%%%%%%%%%%%%%%%%%%%%%%%%%%%%%%%%%%%%%%%%%%%%%%%%%%%%%%%
\section{Explicit Runge-Kutta Scheme}
%%%%%%%%%%%%%%%%%%%%%%%%%%%%%%%%%%%%%%%%%%%%%%%%%%%%%%%%%%%%%%%%%%%%%%%%%%%%

For the fundamental therorem of calculus,
\begin{align}
    y(t_{n+1})
    &=y(t_n)+\int_{t_n}^{t_{n+1}}f(y(\tau),\tau)d\tau
    =y(t_n)+h\int_0^1f(y(t_n+h\tau), t_n+h\tau)d\tau
\end{align}
replace the integral with a quadrature approximation
\begin{align}
    y_{n+1}=y_n+h\Sigma_{i=1}^s b_if(y(t_n+c_ih), t_n+c_ih)
\end{align}
then we have to construct an approximation, denoted by $Y_i\simeq y(t_n+c_ih)$.
With explicit Runge-Kutta methods you construct $Y_i$ using
\begin{align}
    Y_1=y_n,\quad
    f(Y_1, t_n),\quad
    f(Y_2, t_n+hc_2), \quad
    ..., \quad
    f(Y_{i-1}, t_n+hc_{i-1}).
\end{align}
Then
\begin{empheq}[box=\fbox]{align}
    Y_s&=y_n+h\Sigma_{i=1}^{s-1}a_{si}f(Y_i, t_n+h c_i). \\
    y_{n+1}
       &=y_n+h\Sigma_{i=1}^sb_if(Y_i, t_n+h c_i)
\end{empheq}
More
\begin{align}
    y'
    &=f(y,t)
    \\
    y''
    &=\frac{d}{dt}y'
    =\left(\frac{\partial}{\partial t}+f\frac{\partial}{\partial y}\right)f
    =\frac{\partial f}{\partial t}+f\frac{\partial f}{\partial y}
    \\
    y'''
    &=\frac{d}{dt}y''
    =\left(\frac{\partial}{\partial t}+f\frac{\partial}{\partial y}\right)
    \left(\frac{\partial f}{\partial t}+f\frac{\partial f}{\partial y}\right)
    \nonumber\\
    &=\frac{\partial^2f}{\partial t^2}
    +f\frac{\partial^2f}{\partial t\partial y}
    +\frac{\partial f}{\partial t}\frac{\partial f}{\partial y}
    +f\frac{\partial^2f}{\partial y \partial t}
    +f\frac{\partial f}{\partial y}\frac{\partial f}{\partial y}
    +f^2\frac{\partial^2 f}{\partial y^2}
    \nonumber\\
    &=\frac{\partial^2f}{\partial t^2}
    +f^2\frac{\partial^2 f}{\partial y^2}
    +2f\frac{\partial^2f}{\partial t\partial y}
    +\frac{\partial f}{\partial t}\frac{\partial f}{\partial y}
    +f\left(\frac{\partial f}{\partial y}\right)^2
\end{align}

Consider $t_n=0, y(t)=t \Rightarrow y(0)=0, y'=1$. For first order accuracy, need to obtain the exact
solution of $y(t)=t$.
\begin{align}
    Y_i
    &=y_n+h\Sigma_{j=1}^{i-1} a_{ij} \simeq y(t_n+c_ih) = y_n+hc_i, \\
    y_{n+1}
    &=y_n + h\Sigma_{i=1}^sb_i = y_n + h.
    \\~\nonumber\\
    &\Rightarrow
    \left\{
        \begin{matrix}
            &\Sigma_{j=1}^{i-1}a_{ij} = c_i, \\
            &\Sigma_{i=1}^sb_i = 1.
        \end{matrix}
    \right.
    \Rightarrow
    \left\{
        \begin{matrix}
            &a_{21} = c_2
        \end{matrix}
    \right.
\end{align}
where is necessary for first-order accuracy.

For up to the 3-order conditions, it sufices to study the case of \textbf{autonomous} differential equations, $y'=f(y)$. Then
\begin{align}
    y'&=f, \\
    y''&=(f\frac{\partial}{\partial y})f=ff_y, \\
    y'''&=(f\frac{\partial}{\partial y})(f f_y)=ff_yf_y+f^2f_{yy}
\end{align}

\subsection{Taylor expansion of $k_i$ for RK4 up to $\mathcal{O}(h^3)$}

Following the convention of Mongwane, we defined
\begin{align}
    k_i=hf(Y_i,t_n+hc_i)
\end{align}
then
\begin{align}
    k_1&=hy', \\
    k_2&=hy'+\frac{h^2}{2}y''+\frac{h^3}{8}(y'''-f_yy''), \\
    k_3&=hy'+\frac{h^2}{2}y''+\frac{h^3}{8}(y'''+f_yy'').
\end{align}
where $f_yy''\equiv\left(y''\right)^2/y'=\frac{4(k^{(c)}_3-k^{(c)}_2)}{h^3}$,
and $y',y'',y'''$ can be represented with $k^{(c)}_i$ of coarse grid using the dense output formula.

\subsection{Taylor expansion of $Y_i$ for RK4 up to $\mathcal{O}(h^3)$}

Similarly, we can also expand $Y_i$ instead of $k_i$, and we have
\begin{align}
    Y_1&=y_n, \label{eq:Y1} \\
    Y_2&=y_n + \frac{h}{2}y', \label{eq:Y2} \\
    Y_3&=y_n + \frac{h}{2}y' + \frac{h^2}{4}y''+\frac{h^3}{16}(y'''-f_yy'') \label{eq:Y3} \\
    Y_4&=y_n + hy' + \frac{h^2}{2}y''+\frac{h^3}{8}(y'''+f_yy'') \label{eq:Y4}
\end{align}
where $f_yy''\equiv\left(y''\right)^2/y'=\frac{4(k^{(c)}_3-k^{(c)}_2)}{h^3}$,
and $y',y'',y'''$ can be represented with $k^{(c)}_i$ of coarse grid using the dense output formula,
\begin{align}
    y(t_n+\theta h)
    &=y_n+\Sigma_{i=1}^4b_i(\theta)k^{(c)}_i+\mathcal{O}(h^4),
    \label{eq:denseoutput1} \\
    \frac{d^{(m)}}{dt^{(m)}}y(t_n+\theta h)
    &=\frac{1}{h^m}\Sigma_{i=1}^sk^{(c)}_i
    \frac{d^{(m)}}{d\theta^{(m)}}b_i(\theta)+\mathcal{O}(h^{4-m}).
    \label{eq:denseoutput2}
\end{align}
where
$b_1(\theta)=\theta-\frac{3}{2}\theta^2+\frac{2}{3}\theta^3$,
$b_2(\theta)=b_3(\theta)=\theta^2-\frac{2}{3}\theta^3$,
$b_4(\theta)=-\frac{1}{2}\theta^2+\frac{2}{3}\theta^3$.

\subsubsection{Pseudo code}
\begin{enumerate}
    \item integrate coarse grid from $t_n$ to $t_n+h^{(c)}$ and store $k^{(c)}_i$
        somewhere,
    \item interpolate in time for $y,y',y'',y'''$ using
        \eqref{eq:denseoutput1}-\eqref{eq:denseoutput2}
        \begin{enumerate}[label=(\alph*)]
            \item at $\theta=0.0$ for the first fine step,
            \item at $\theta=0.5$ for the second fine step,
        \end{enumerate}
    \item calculate $Y_i$ for the fine grid using \eqref{eq:Y1}-\eqref{eq:Y4}.
    \item interpolate in space to fill $Y_i$ in the fine ghost points.
\end{enumerate}



%\bibliographystyle{unsrt}
%\bibliography{references}

\end{document}
