\documentclass[prd,aps,a4paper,superscriptaddress,onecolumn,footinbib]{revtex4}
\usepackage{graphicx}
\usepackage{color}
\usepackage{dcolumn}
\usepackage{bm}
\usepackage{upgreek}
\usepackage{slashed}
\usepackage{amsmath}
\usepackage{empheq}
\usepackage{latexsym}
\usepackage{amssymb}
\usepackage{amsfonts}
\usepackage{dsfont}
\usepackage{listings}
\usepackage{xcolor}
\usepackage{ulem}
\usepackage{cancel}
\usepackage{mathtools}
\usepackage{enumitem}
\setcounter{MaxMatrixCols}{18}



%%
\newcommand{\mytext}{}
\newcommand{\LJ}[1]{{\textcolor{blue}{\mytext{LJ: #1}} }}
\newcommand{\WT}[1]{{\textcolor{red}{\mytext{WT: #1}} }}
%%



\begin{document}
\title{Notes on Implementing MC's Method}

\author{Liwei~Ji}
\email{ljsma@rit.edu}
\affiliation{Center for Computational Relativity and Gravitation,
Rochester Institute of Technology, Rochester, New York 14623, USA}

\maketitle

\tableofcontents

\lstset{numbers=left,
  numberstyle= \tiny,
  keywordstyle= \color{ blue!70},commentstyle=\color{red!50!green!50!blue!50},
  frame=shadowbox,
  rulesepcolor= \color{ red!20!green!20!blue!20}
}


%%%%%%%%%%%%%%%%%%%%%%%%%%%%%%%%%%%%%%%%%%%%%%%%%%%%%%%%%%%%%%%%%%%%%%%%%%%%
\section{Explicit Runge-Kutta Scheme}
%%%%%%%%%%%%%%%%%%%%%%%%%%%%%%%%%%%%%%%%%%%%%%%%%%%%%%%%%%%%%%%%%%%%%%%%%%%%
To integrate the ODE
\begin{align}
    \frac{dy}{dt}=f(y,t),
\end{align}
we start from the fundamental therorem of calculus
\begin{align}
    y(t_{n+1})
    &=y(t_n)+\int_{t_n}^{t_{n+1}}f(y(\tau),\tau)d\tau
    =y(t_n)+h\int_0^1f(y(t_n+h\tau), t_n+h\tau)d\tau.
\end{align}
We can replace the integral with a quadrature approximation
\begin{align}
    y_{n+1}=y_n+h\Sigma_{i=1}^s b_if(y(t_n+c_ih), t_n+c_ih),
\end{align}
where we have to construct an approximation, denoted by $Y_i\simeq y(t_n+c_ih)$.
With explicit Runge-Kutta method we construct $Y_i$ using
\begin{align}
    y_n,\quad
    f(Y_1, t_n+hc_1),\quad
    f(Y_2, t_n+hc_2), \quad
       ..., \quad
    f(Y_{i-1}, t_n+hc_{i-1}).
\end{align}
Then the explicit Runge-Kutta method can be sum up as the following,
\begin{empheq}[box=\fbox]{align}
    Y_s&=y_n+h\Sigma_{i=1}^{s-1}a_{si}f(Y_i, t_n+h c_i), \label{eq:RK_Y} \\
    y_{n+1}
       &=y_n+h\Sigma_{i=1}^sb_if(Y_i, t_n+h c_i). \label{eq:RK_y}
\end{empheq}
We can also introduce the intermediate slopes as $k_i:=f(Y_i,t_n+hc_i)$, then
\begin{align}
    y_{n+1}&=y_n+h\Sigma_{i=1}^sb_ik_i.
\end{align}
Further more for the chain rule,
\begin{align}
    y'
    &=f(y,t)
    \\
    y''
    &=\frac{d}{dt}y'
    =\left(\frac{\partial}{\partial t}+f\frac{\partial}{\partial y}\right)f
    =\frac{\partial f}{\partial t}+f\frac{\partial f}{\partial y}
    \\
    y'''
    &=\frac{d}{dt}y''
    =\left(\frac{\partial}{\partial t}+f\frac{\partial}{\partial y}\right)
    \left(\frac{\partial f}{\partial t}+f\frac{\partial f}{\partial y}\right)
    \nonumber\\
    &=\frac{\partial^2f}{\partial t^2}
    +f\frac{\partial^2f}{\partial t\partial y}
    +\frac{\partial f}{\partial t}\frac{\partial f}{\partial y}
    +f\frac{\partial^2f}{\partial y \partial t}
    +f\frac{\partial f}{\partial y}\frac{\partial f}{\partial y}
    +f^2\frac{\partial^2 f}{\partial y^2}
    \nonumber\\
    &=\frac{\partial^2f}{\partial t^2}
    +2f\frac{\partial^2f}{\partial t\partial y}
    +\frac{\partial f}{\partial t}\frac{\partial f}{\partial y}
    +f^2\frac{\partial^2 f}{\partial y^2}
    +f\left(\frac{\partial f}{\partial y}\right)^2
\end{align}

Consider the case
$t_n=0,\; y(t)=t \Rightarrow y(0)=0, y'=1$,
for the first order accuracy, the above scheme need to obtain the exact
solution of $y(t)=t$.
\begin{align}
    Y_i
    &=y_n+h\Sigma_{j=1}^{i-1} a_{ij} \simeq y(t_n+c_ih) = y_n+hc_i, \\
    y_{n+1}
    &=y_n + h\Sigma_{i=1}^sb_i = y_n + h.
    \\~\nonumber\\
    &\Rightarrow
    \left\{
        \begin{matrix}
            &\Sigma_{j=1}^{i-1}a_{ij} = c_i, \\
            &\Sigma_{i=1}^sb_i = 1,
        \end{matrix}
    \right.
\end{align}
$a_{21}=c_2$ for example.

For up to the 3-order conditions, it sufices to study the case of \textbf{autonomous} differential equations, $y'=f(y)$, where the chain rule become
\begin{align}
    y'&=f, \\
    y''&=(f\frac{\partial}{\partial y})f=ff_y, \\
    y'''&=(f\frac{\partial}{\partial y})(f f_y)=ff_y^2+f^2f_{yy}
\end{align}

\subsection{Dense Output}

Consider formulas of the form for a 4-stage Runge-Kutta method \cite{alexander1990solving}
\begin{align}
    u(\theta)
    &=y_n+h\Sigma_{i=1}^4b_i(\theta)k_i.
\end{align}
where $k_i$ is defined below eq.~\eqref{eq:RK_y},
and $b_i(\theta)$ are polynomicals to be determined such that
$u(\theta)-y(x_0+\theta h)=\mathcal{O}(h^{p^*+1})$.
We write the polynomials $b_j(\theta)$ as
\begin{align}
    b_j(\theta)=\Sigma_{q=1}^{p^*}b_{jq}\theta^q.
\end{align}
For the 4-stage RK4 with $p^*=3$ the order condition produce a unique solution
\begin{align}
    b_1(\theta)&=\theta-\frac{3}{2}\theta^2+\frac{2}{3}\theta^3, \\
    b_2(\theta)&=b_3(\theta)=\theta^2-\frac{2}{3}\theta^3, \\
    b_4(\theta)&=-\frac{1}{2}\theta^2+\frac{2}{3}\theta^3.
\end{align}

The dense output formula can be summarized as \cite{mongwane2015toward}
\begin{empheq}[box=\fbox]{align}
    y(t_n+\theta h)
    &=y_n+h\Sigma_{i=1}^4b_i(\theta)k_i+\mathcal{O}(h^4),
    \label{eq:denseoutput1} \\
    \frac{d^{(m)}}{dt^{(m)}}y(t_n+\theta h)
    &=\frac{1}{h^{(m-1)}}\Sigma_{i=1}^sk_i
    \frac{d^{(m)}}{d\theta^{(m)}}b_i(\theta)+\mathcal{O}(h^{4-m}).
    \label{eq:denseoutput2}
\end{empheq}

\subsection{Taylor expansion of $k_i$ for RK4 up to $\mathcal{O}(h^2)$}

The taylor expansion of $k_i$ around $t_n$ are \cite{mongwane2015toward}
\begin{align}
    k_1&=y', \\
    k_2&=y'+\frac{h}{2}y''+\frac{h^2}{8}(y'''-f_yy''), \\
    k_3&=y'+\frac{h}{2}y''+\frac{h^2}{8}(y'''+f_yy'').
\end{align}
where
$f_yy''\equiv\left(y''\right)^2/y'=\frac{4(k^{(c)}_3-k^{(c)}_2)}{h^2_{(c)}}$,
and $y',y'',y'''$ can be represented with $k^{(c)}_i$ of coarse grid using the dense output formula \eqref{eq:denseoutput1}-\eqref{eq:denseoutput2}.

\subsection{Taylor expansion of $Y_i$ for RK4 up to $\mathcal{O}(h^3)$}

Similarly, we can also expand $Y_i$ instead of $k_i$ \cite{mccorquodale2011high}
\begin{align}
    Y_1&=y_n, \label{eq:Y1} \\
    Y_2&=y_n + \frac{h}{2}y', \label{eq:Y2} \\
    Y_3&=y_n + \frac{h}{2}y' + \frac{h^2}{4}y''+\frac{h^3}{16}(y'''-f_yy'') \label{eq:Y3} \\
    Y_4&=y_n + hy' + \frac{h^2}{2}y''+\frac{h^3}{8}(y'''+f_yy'') \label{eq:Y4}
\end{align}
where $f_yy''\equiv\left(y''\right)^2/y'=\frac{4(k^{(c)}_3-k^{(c)}_2)}{h^2_{(c)}}$,
and $y',y'',y'''$ can be represented with $k^{(c)}_i$ of coarse grid using the dense output formula \eqref{eq:denseoutput1}-\eqref{eq:denseoutput2}.

\subsubsection{Pseudocode of MC's method}
\begin{enumerate}
    \item Integrate coarse grid from $t_n$ to $t_n+h^{(c)}$ and store $k^{(c)}_i$
        somewhere.
    \item Interpolate in time for $y,y',y'',y'''$ using
        \eqref{eq:denseoutput1}-\eqref{eq:denseoutput2} with stored $k_i^{(c)}$
        \begin{enumerate}[label=(\alph*)]
            \item at $\theta=0.0$ for the first fine step,
            \item at $\theta=0.5$ for the second fine step.
        \end{enumerate}
    \item Calculate $Y_i$ for the first and second fine steps using \eqref{eq:Y1}-\eqref{eq:Y4}.
    \item Interpolate in space to fill $Y_i$ in the fine ghost points.
    \item Repeat.
\end{enumerate}



\bibliographystyle{unsrt}
\bibliography{refs}

\end{document}
