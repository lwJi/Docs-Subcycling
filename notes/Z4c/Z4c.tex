\documentclass[prd,aps,a4paper,superscriptaddress,onecolumn,footinbib]{revtex4}
\usepackage{graphicx}
\usepackage{color}
\usepackage{dcolumn}
\usepackage{bm}
\usepackage{upgreek}
\usepackage{slashed}
\usepackage{amsmath}
\usepackage{empheq}
\usepackage{latexsym}
\usepackage{amssymb}
\usepackage{amsfonts}
\usepackage{dsfont}
\usepackage{listings}
\usepackage{xcolor}
\usepackage{ulem}
\usepackage{cancel}
\usepackage{mathtools}
\usepackage{enumitem}
\setcounter{MaxMatrixCols}{18}



%%
\newcommand{\mytext}{}
\newcommand{\LJ}[1]{{\textcolor{blue}{\mytext{LJ: #1}} }}
\newcommand{\WT}[1]{{\textcolor{red}{\mytext{WT: #1}} }}
%%



\begin{document}
\title{Notes on Z4c}

\author{Liwei~Ji}
\email{ljsma@rit.edu}
\affiliation{Center for Computational Relativity and Gravitation,
Rochester Institute of Technology, Rochester, New York 14623, USA}

\maketitle

\tableofcontents

\lstset{numbers=left,
  numberstyle= \tiny,
  keywordstyle= \color{ blue!70},commentstyle=\color{red!50!green!50!blue!50},
  frame=shadowbox,
  rulesepcolor= \color{ red!20!green!20!blue!20}
}


%%%%%%%%%%%%%%%%%%%%%%%%%%%%%%%%%%%%%%%%%%%%%%%%%%%%%%%%%%%%%%%%%%%%%%%%%%%%
\section{Derivation}
%%%%%%%%%%%%%%%%%%%%%%%%%%%%%%%%%%%%%%%%%%%%%%%%%%%%%%%%%%%%%%%%%%%%%%%%%%%%

\begin{itemize}
\item $D_iD_j\alpha$:
  \begin{align}
    \Gamma^{k}{}_{ij}
    &=\frac{1}{2}\gamma^{kl}(\partial_i\gamma_{jl}+\partial_j\gamma_{li}-\partial_l\gamma_{ij})
    \\
    &=\frac{1}{2}\tilde\gamma^{kl}
    \left[
      (\partial_i\tilde\gamma_{jl}-\partial_i\ln\chi\tilde\gamma_{jl})+
      (\partial_j\tilde\gamma_{li}-\partial_j\ln\chi\tilde\gamma_{li})-
      (\partial_l\tilde\gamma_{ij}-\partial_l\ln\chi\tilde\gamma_{ij})
    \right]
    \\
    &=\tilde\Gamma^k{}_{ij}
    -\frac{1}{2}
    (\partial_i\ln\chi\delta^k{}_j+\partial_j\ln\chi\delta^k{}_i
    -\tilde\gamma_{ij}\tilde\gamma^{kl}\partial_l\ln\chi)
  \end{align}
  where
  $\partial_l\gamma_{ij}=\partial_l(\chi^{-1}\tilde\gamma_{ij})
  =\chi^{-1}(\partial_l\tilde{\gamma}_{ij}-\chi^{-1}\partial_l\chi\tilde\gamma_{ij})
  =\chi^{-1}(\partial_l\tilde{\gamma}_{ij}-\partial_l\ln\chi\tilde\gamma_{ij})$.
  Then,
  \begin{align}
    D_iD_j\alpha
    &=\partial_i\partial_j\alpha-\Gamma^k{}_{ij}\partial_k\alpha \\
    &=\partial_i\partial_j\alpha-
    \left[
      \tilde\Gamma^k{}_{ij}
      -\frac{1}{2}
      (\partial_i\ln\chi\delta^k{}_j+\partial_j\ln\chi\delta^k{}_i
      -\tilde\gamma_{ij}\tilde\gamma^{kl}\partial_l\ln\chi)
    \right]\partial_k\alpha \\
    &=\partial_i\partial_j\alpha-
    \tilde\Gamma^k{}_{ij}\partial_k\alpha
    +\frac{1}{2}
    (\partial_i\ln\chi\partial_j\alpha
    +\partial_j\ln\chi\partial_i\alpha
    -\tilde\gamma_{ij}\tilde\gamma^{kl}\partial_l\ln\chi\partial_k\alpha) \\
    &=\partial_i\partial_j\alpha-
    \tilde\Gamma^k{}_{ij}\partial_k\alpha
    +\partial_{(i}\ln\chi\partial_{j)}\alpha
    -\frac{1}{2}\tilde\gamma_{ij}\tilde\gamma^{kl}\partial_l\ln\chi\partial_k\alpha
  \end{align}
\end{itemize}


\section{More}
\cite{alexander1990solving}



\bibliographystyle{unsrt}
\bibliography{refs}

\end{document}
