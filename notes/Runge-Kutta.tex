\documentclass[prd,aps,a4paper,superscriptaddress,onecolumn,footinbib]{revtex4}
\usepackage{graphicx}
\usepackage{color}
\usepackage{dcolumn}
\usepackage{bm}
\usepackage{upgreek}
\usepackage{slashed}
\usepackage{amsmath}
\usepackage{latexsym}
\usepackage{amssymb}
\usepackage{amsfonts}
\usepackage{dsfont}
\usepackage{listings}
\usepackage{xcolor}
\usepackage{ulem}
\usepackage{cancel}
\usepackage{mathtools}
\setcounter{MaxMatrixCols}{18}



%%
\newcommand{\mytext}{}
\newcommand{\LJ}[1]{{\textcolor{blue}{\mytext{LJ: #1}} }}
\newcommand{\WT}[1]{{\textcolor{red}{\mytext{WT: #1}} }}
%%



\begin{document}
\title{Notes on Runge-Kutta method}

\author{Liwei~Ji}
\email{ljsma@rit.edu}
\affiliation{Center for Computational Relativity and Gravitation,
Rochester Institute of Technology, Rochester, New York 14623, USA}

\maketitle

\tableofcontents

\lstset{numbers=left,
  numberstyle= \tiny,
  keywordstyle= \color{ blue!70},commentstyle=\color{red!50!green!50!blue!50},
  frame=shadowbox,
  rulesepcolor= \color{ red!20!green!20!blue!20}
}


%%%%%%%%%%%%%%%%%%%%%%%%%%%%%%%%%%%%%%%%%%%%%%%%%%%%%%%%%%%%%%%%%%%%%%%%%%%%
\section{Explicit Runge-Kutta Scheme}
%%%%%%%%%%%%%%%%%%%%%%%%%%%%%%%%%%%%%%%%%%%%%%%%%%%%%%%%%%%%%%%%%%%%%%%%%%%%

\begin{align}
    y'
    &=f(y,t)
    \\
    y''
    &=\frac{d}{dt}y'
    =\left(\frac{\partial}{\partial t}+f\frac{\partial}{\partial y}\right)f
    =\frac{\partial f}{\partial t}+f\frac{\partial f}{\partial y}
\end{align}

\subsection{Two-stage, second-order RK}

\begin{align}
    Y_1&=y_n \\
    Y_2&=y_n+h a_{21}f(Y_1, t_n) \\
    Y_3&=y_n+h b_1f(Y_1, t_n)+h b_2f(Y_2, t_n+c_2h)
\end{align}

Taylar expand about $(y_n, t_n)$:
\begin{align}
    y_{n+1}
    &=y_n + h b_1 f(y_n, t_n) + h b_2
    \left[
        f + h
        \left(
            a_{21} \frac{\partial f}{\partial y} f + c_2\frac{\partial f}{\partial t}
        \right)
    \right]
    + \mathcal{O}(h^3)
    \\
    &=y_n + h(b_1+b_2)f + h^2 b_2
    \left(a_{21}\frac{\partial f}{\partial y}f + c_2\frac{\partial f}{\partial t}\right)
    + \mathcal{O}(h^3)
\end{align}
since
\begin{align}
    f(Y_2, t_n+c_2h)
    &=f(y_n+h a_{21}f(Y_1, t_n), t_n+c_2h) \\
    &=f(y_n, t_n) + h
    \left[
        \frac{\partial f}{\partial y}(y_n, t_n) a_{21}f(y_n, t_n)
        +\frac{\partial f}{\partial t}(y_n, t_n) c_2
    \right]
    + \mathcal{O}(h^2)
\end{align}

Compare to the Tayler expansion
\begin{align}
    y(t_n+h)
    &=y(t_n) + hy'(t_n) + \frac{h^2}{2}y''(t_n)
    + \mathcal{O}(h^3)
    \\
    &=y_n + h f + \frac{h^2}{2} \left(\frac{\partial f}{\partial t}+f\frac{\partial f}{\partial y}\right)
    + \mathcal{O}(h^3)
\end{align}
we have the order condition
\begin{align}
    &b_1+b_2 = 1 \\
    &b_2a_{21} = \frac{1}{2} \\
    &b_2c_2 = \frac{1}{2}
\end{align}

\subsection{Three-stage, third-order RK}






%\bibliographystyle{unsrt}
%\bibliography{references}

\end{document}
