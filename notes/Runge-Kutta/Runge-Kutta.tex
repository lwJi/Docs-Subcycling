\documentclass[prd,aps,a4paper,superscriptaddress,onecolumn,footinbib]{revtex4}
\usepackage{graphicx}
\usepackage{color}
\usepackage{dcolumn}
\usepackage{bm}
\usepackage{upgreek}
\usepackage{slashed}
\usepackage{amsmath}
\usepackage{latexsym}
\usepackage{amssymb}
\usepackage{amsfonts}
\usepackage{dsfont}
\usepackage{listings}
\usepackage{xcolor}
\usepackage{ulem}
\usepackage{cancel}
\usepackage{mathtools}
\setcounter{MaxMatrixCols}{18}



%%
\newcommand{\mytext}{}
\newcommand{\LJ}[1]{{\textcolor{blue}{\mytext{LJ: #1}} }}
\newcommand{\WT}[1]{{\textcolor{red}{\mytext{WT: #1}} }}
%%



\begin{document}
\title{Notes on Runge-Kutta method}

\author{Liwei~Ji}
\email{ljsma@rit.edu}
\affiliation{Center for Computational Relativity and Gravitation,
Rochester Institute of Technology, Rochester, New York 14623, USA}

\maketitle

\tableofcontents

\lstset{numbers=left,
  numberstyle= \tiny,
  keywordstyle= \color{ blue!70},commentstyle=\color{red!50!green!50!blue!50},
  frame=shadowbox,
  rulesepcolor= \color{ red!20!green!20!blue!20}
}


%%%%%%%%%%%%%%%%%%%%%%%%%%%%%%%%%%%%%%%%%%%%%%%%%%%%%%%%%%%%%%%%%%%%%%%%%%%%
\section{Explicit Runge-Kutta Scheme}
%%%%%%%%%%%%%%%%%%%%%%%%%%%%%%%%%%%%%%%%%%%%%%%%%%%%%%%%%%%%%%%%%%%%%%%%%%%%

For the fundamental therorem of calculus,
\begin{align}
    y(t_{n+1})
    &=y(t_n)+\int_{t_n}^{t_{n+1}}f(y(\tau),\tau)d\tau
    =y(t_n)+h\int_0^1f(y(t_n+h\tau), t_n+h\tau)d\tau
\end{align}

replace the integral with a quadrature approximation
\begin{align}
    y_{n+1}=y_n+h\Sigma_{i=1}^s b_if(y(t_n+c_ih), t_n+c_ih)
\end{align}
then we have to construct an approximation, denoted by $Y_i\simeq y(t_n+c_ih)$.
With explicit Runge-Kutta methods you construct $Y_i$ using
\begin{align}
    Y_1=y_n,\quad
    f(Y_1, t_n),\quad
    f(Y_2, t_n+hc_2), \quad
    ..., \quad
    f(Y_{i-1}, t_n+hc_{i-1}).
\end{align}
Then
\begin{align}
    Y_1&=y_n, \\
    Y_2&=y_n+h a_{21}f(Y_1,t_n), \\
    Y_3&=y_n+h a_{31}f(Y_1,t_n) + h a_{32}f(Y_2, t_n+h c_2), \\
       &... \\
    Y_s&=y_n+h\Sigma_{i=1}^{s-1}a_{si}f(Y_i, t_n+h c_i). \\
    y_{n+1}
       &=y_n+h\Sigma_{i=1}^sb_if(Y_i, t_n+h c_i)
\end{align}

More
\begin{align}
    y'
    &=f(y,t)
    \\
    y''
    &=\frac{d}{dt}y'
    =\left(\frac{\partial}{\partial t}+f\frac{\partial}{\partial y}\right)f
    =\frac{\partial f}{\partial t}+f\frac{\partial f}{\partial y}
    \\
    y'''
    &=\frac{d}{dt}y''
    =\left(\frac{\partial}{\partial t}+f\frac{\partial}{\partial y}\right)
    \left(\frac{\partial f}{\partial t}+f\frac{\partial f}{\partial y}\right)
    \\
    &=\frac{\partial^2f}{\partial t^2}
    +f\frac{\partial^2f}{\partial t\partial y}
    +\frac{\partial f}{\partial t}\frac{\partial f}{\partial y}
    +f\frac{\partial^2f}{\partial y \partial t}
    +f\frac{\partial f}{\partial y}\frac{\partial f}{\partial y}
    +f^2\frac{\partial^2 f}{\partial y^2}
    \\
    &=\frac{\partial^2f}{\partial t^2}
    +f^2\frac{\partial^2 f}{\partial y^2}
    +2f\frac{\partial^2f}{\partial t\partial y}
    +\frac{\partial f}{\partial t}\frac{\partial f}{\partial y}
    +f\left(\frac{\partial f}{\partial y}\right)^2
\end{align}

\subsection{Two-stage, second-order RK}

\begin{align}
    Y_1&=y_n \\
    Y_2&=y_n+h a_{21}f(Y_1, t_n) \\
    y_{n+1}
       &=y_n+h b_1f(Y_1, t_n)+h b_2f(Y_2, t_n+c_2h)
\end{align}

Taylar expand about $(y_n, t_n)$:
\begin{align}
    y_{n+1}
    &=y_n + h b_1 f(y_n, t_n) + h b_2
    \left[
        f + h
        \left(
            a_{21} \frac{\partial f}{\partial y} f + c_2\frac{\partial f}{\partial t}
        \right)
    \right]
    + \mathcal{O}(h^3)
    \\
    &=y_n + h(b_1+b_2)f + h^2 b_2
    \left(a_{21}\frac{\partial f}{\partial y}f + c_2\frac{\partial f}{\partial t}\right)
    + \mathcal{O}(h^3)
\end{align}
since
\begin{align}
    f(Y_2, t_n+c_2h)
    &=f(y_n+h a_{21}f(Y_1, t_n), t_n+c_2h) \\
    &=f(y_n, t_n) + h
    \left[
        \frac{\partial f}{\partial y}(y_n, t_n) a_{21}f(y_n, t_n)
        +\frac{\partial f}{\partial t}(y_n, t_n) c_2
    \right]
    + \mathcal{O}(h^2)
\end{align}

Compare to the Tayler expansion
\begin{align}
    y(t_n+h)
    &=y(t_n) + hy'(t_n) + \frac{h^2}{2}y''(t_n)
    + \mathcal{O}(h^3)
    \\
    &=y_n + h f + \frac{h^2}{2} \left(\frac{\partial f}{\partial t}+f\frac{\partial f}{\partial y}\right)
    + \mathcal{O}(h^3)
\end{align}
we have the order condition
\begin{align}
    &b_1+b_2 = 1 \\
    &b_2a_{21} = \frac{1}{2} \\
    &b_2c_2 = \frac{1}{2}
\end{align}

\subsection{Three-stage, third-order RK}

\begin{align}
    Y_i
    &= y_n + h \Sigma_{j=1}^{i-1}a_{ij}f(Y_j, t_c+c_jh)
    \\
    y_{n+1}
    &=y_n + h \Sigma_{i=1}^{s}b_i f(Y_i, t_n+c_ih)
\end{align}

Consider $t_n=0, y(t)=t \Rightarrow y(0)=0, y'=1$. For first order accuracy, need to obtain the exact
solution of $y(t)=t$.
\begin{align}
    Y_i
    &=y_n+h\Sigma_{j=1}^{i-1} a_{ij} \simeq y(t_n+c_ih) = y_n+hc_i, \\
    y_{n+1}
    &=y_n + h\Sigma_{i=1}^sb_i = y_n + h.
\end{align}
\begin{align}
    &\Rightarrow
    \left\{
        \begin{matrix}
            &\Sigma_{j=1}^{i-1}a_{ij} = c_i, \\
            &\Sigma_{i=1}^sb_i = 1.
        \end{matrix}
    \right.
    \Rightarrow
    \left\{
        \begin{matrix}
            &a_{21} = c_2
        \end{matrix}
    \right.
\end{align}
where is necessary for first-order accuracy.

For up to the 3-order conditions, it sufices to study the case of autonomous differential
equations, $y'=f(y)$.

\begin{align}
    Y_1&=y_n, \\
    Y_2&=y_n+hc_2f, \\
    Y_3&=y_n+h(c_3-a_{32})f(Y_1) + ha_{32}f(Y_2) \\
       &=y_n+h(c_3-a_{32})f + ha_{32}
       \left(
           f+hc_2f f_y+\frac{(hc_2f)^2}{2}f_{yy}+\mathcal{O}(h^3)
        \right) \\
       &=y_n + hc_3f + h^2a_{32}c_2f_yf+\frac{h^3a_{32}c_2^2}{2}f_{yy}f^2+\mathcal{O}(h^4)
\end{align}
where we have used
\begin{align}
    f(Y_2)&=f(y_n+hc_2f)=f+hc_2f f_y+\frac{(hc_2f)^2}{2}f_{yy}+\mathcal{O}(h^3)\\
    f(Y_3)&=f(y_n + hc_3f + h^2a_{32}c_2f_yf + \mathcal{O}(h^3)) \\
          &=f + f_y
          \left(
              hc_3f+h^2a_{32}c_2f_yf + \mathcal{O}(h^3)
          \right)
          +\frac{1}{2}f_{yy}
          \left(
              hc_3f+h^2a_{32}c_2f_yf + \mathcal{O}(h^3)
          \right)^2
          + \mathcal{O}(h^3)
          \\
          &=f + hc_3ff_y + h^2a_{32}c_2f_y^2 f
          + \frac{1}{2}f_{yy}(h^2c_3^2f^2)
          + \mathcal{O}(h^3)
          \\
          &=f + hc_3ff_y
          + h^2(a_{32}c_2f_y^2 f + \frac{1}{2}c_3^2f^2f_{yy})
          + \mathcal{O}(h^3)
\end{align}
then
\begin{align}
    y_{n+1}
    &=y_n + hb_1f + h b_2 \left(f+hc_2f f_y+\frac{(hc_2f)^2}{2}f_{yy}\right) \\
    &\qquad\qquad\qquad
    + h b_3
    \left(
        f + hc_3ff_y + h^2(a_{32}c_2f_y^2 f + \frac{1}{2}c_3^2f^2f_{yy})
    \right)
    + \mathcal{O}(h^4) \\
    &=y_n + h(b_1+b_2+b_3)f
    + h^2(c_2b_2+c_3b_3)f_yf
    + h^3(\frac{1}{2}(b_2c_3^2+b_3c_3^2)f_{yy}f^2 + b_3a_{32}c_2f_y^2f)
    + \mathcal{O}(h^4)
\end{align}

Tayler expansion
\begin{align}
    y(t_n+h)
    &=y_n + hf + \frac{h^2}{2}(f f_y)
    + \frac{h^3}{6}
    \left(
        ff_y^2+f^2f_{yy}
    \right)
    + \mathcal{O}(h^4)
\end{align}
where we have used
\begin{align}
    y'=f, \quad
    y''=(f\frac{\partial}{\partial y})f=ff_y, \quad
    y'''=(f\frac{\partial}{\partial y})(f f_y)=ff_yf_y+f^2f_{yy}
\end{align}

Compare the above two equations, we got the order condition
\begin{align}
    &b_1+b_2+b_3 = 1, \\
    &b_2c_2+b_3c_3 = \frac{1}{2}, \\
    &\frac{1}{2}(b_2c_2^2+b_3c_3^3)=\frac{1}{6}, \\
    &b_ca_{32}c_2 = \frac{1}{6}.
\end{align}






%\bibliographystyle{unsrt}
%\bibliography{references}

\end{document}
